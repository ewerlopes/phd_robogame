\chapter{Post-match questionnaire}\label{app:questionnaire}

\section{English Version}

\subsection*{Who are you?}

%\Qitem{ \Qq{Nome}: \Qline{8cm} }

\Qitem{ \Qq{Age (completed years):} \Qline{1.5cm} }

\Qitem{ \Qq{Gender:} \hskip0.4cm \QO{} M \hskip0.9cm  \QO{} F}

\Qitem[a]{ \Qq{Have you already had a direct experience with robots?} \hskip0.4cm \QO{}
No \hskip0.9cm \QO{} Yes }
\Qitem[b]{ \Qq{If you answered ``Yes" to the previous question, please describe your experience:} \Qlines{4}}
\Qitem{ \Qq{What do you usually play?} \newline \QO{} Videogames \hskip0.7cm \QO{} Board games \hskip0.7cm \QO{} Cards \hskip0.7cm \QO{} Sport \hskip0.7cm \QO{} Other \Qline{5.1 cm}}
%\Qitem{ \Qq{Do you like having an option?}
%\begin{Qlist}
%\item Yes, this meets my need for autonomy.
%\item No, I'd rather have someone else decide for me.
%\item Not sure.
%\end{Qlist}
%}
%
%\Qitem{ \Qq{What kind of music do you like best?}
%\begin{Qlist}
%\item Johann Sebastian Bach
%\item Jazz
%\item The Beatles
%\item Other: \Qline{4cm}
%\end{Qlist}
%}
%
%\minisec{Please evaluate the following composers}
%\vskip.5em
%
%\Qitem[a]{ \Qq{Mozart} \Qtab{3cm}{horrible \Qrating{5} fantastic}}
%
%\Qitem[b]{ \Qq{Beethoven} \Qtab{3cm}{horrible \Qrating{5} fantastic}}
%
%\Qitem[c]{ \Qq{Johann S. Bach} \Qtab{3cm}{horrible \Qrating{5}
%fantastic}}
%
%\Qitem[d]{ \Qq{John Lennon}\Qtab{2.5cm}{\parbox[t]{3.3cm}{\raggedleft
%Uh, well, I do not like his music very much}
%\Qrating{7} \parbox[t]{3cm}{Oh, yes, you know, really great
%stuff}}}
%
%\Qitem[e]{ \Qq{Elvis}\Qtab{2.5cm}{\parbox[t]{3.3cm}{\raggedleft Uh,
%well, I do not like his music very much}
%\Qrating{7} \parbox[t]{3cm}{Oh, yes, you know, really great
%stuff}}}

\subsection*{With respect to the game just played, state how much do you agree with the following statements, where: 1 = ``strongly disagree'', 3 = ``indifferent'' e 5 = ``strongly agree''}

\Qitem{ \Qq{I had fun} \newline \QO{} 1 \hskip0.9cm \QO{} 2 \hskip0.9cm \QO{} 3 \hskip0.9cm \QO{} 4 \hskip0.9cm \QO{} 5}
\Qitem{ \Qq{I wanted to play} \newline \QO{} 1 \hskip0.9cm \QO{} 2 \hskip0.9cm \QO{} 3 \hskip0.9cm \QO{} 4 \hskip0.9cm \QO{} 5}
\Qitem{ \Qq{While playing, I paid attention to the robot} \newline \QO{} 1 \hskip0.9cm \QO{} 2 \hskip0.9cm \QO{} 3 \hskip0.9cm \QO{} 4 \hskip0.9cm \QO{} 5}
\Qitem{ \Qq{The robot was paying attention to what I was doing} \newline \QO{} 1 \hskip0.9cm \QO{} 2 \hskip0.9cm \QO{} 3 \hskip0.9cm \QO{} 4 \hskip0.9cm \QO{} 5}
\Qitem{ \Qq{The robot wanted to win} \newline \QO{} 1 \hskip0.9cm \QO{} 2 \hskip0.9cm \QO{} 3 \hskip0.9cm \QO{} 4 \hskip0.9cm \QO{} 5}
\Qitem{ \Qq{I was scared by the robot} \newline \QO{} 1 \hskip0.9cm \QO{} 2 \hskip0.9cm \QO{} 3 \hskip0.9cm \QO{} 4 \hskip0.9cm \QO{} 5}
\Qitem{ \Qq{The game lasted too few} \newline \QO{} 1 \hskip0.9cm \QO{} 2 \hskip0.9cm \QO{} 3 \hskip0.9cm \QO{} 4 \hskip0.9cm \QO{} 5}
\Qitem{ \Qq{I understood immediately the game rules} \newline \QO{} 1 \hskip0.9cm \QO{} 2 \hskip0.9cm \QO{} 3 \hskip0.9cm \QO{} 4 \hskip0.9cm \QO{} 5}
\Qitem{ \Qq{I changed my playing style during the game} \newline \QO{} 1 \hskip0.9cm \QO{} 2 \hskip0.9cm \QO{} 3 \hskip0.9cm \QO{} 4 \hskip0.9cm \QO{} 5}
\Qitem{ \Qq{The robot did some feints} \newline \QO{} 1 \hskip0.9cm \QO{} 2 \hskip0.9cm \QO{} 3 \hskip0.9cm \QO{} 4 \hskip0.9cm \QO{} 5}
\Qitem{ \Qq{I would enjoy more if the robot did some feints} \newline \QO{} 1 \hskip0.9cm \QO{} 2 \hskip0.9cm \QO{} 3 \hskip0.9cm \QO{} 4 \hskip0.9cm \QO{} 5}
\Qitem{ \Qq{It was easy to understand what the robot would have done} \newline \QO{} 1 \hskip0.9cm \QO{} 2 \hskip0.9cm \QO{} 3 \hskip0.9cm \QO{} 4 \hskip0.9cm \QO{} 5}

%
%\Qitem{\Qq{Do you like the style so far?} \Qtab{5.5cm}{\QO{} Yes
%\hskip0.5cm \QO{} No}}
%
%\Qitem{\Qq{Is it really worth the ink?} \Qtab{5.5cm}{\QO{} Guess so.
%\hskip0.5cm \QO{} Probably not. \hskip0.5cm \QO{} Don't know.}}
%
%\Qitem{\Qq{How does this item look different from the previous one?}
%\Qtab{10.5cm}{\QO{}\Qtab{0.6cm}{Oh. Does it?}}\par
%\Qtab{10.5cm}{\QO{}\Qtab{0.6cm}{No clue. I just can't figure it out.
%So sorry.}}\par
%\Qtab{10.5cm}{\QO{}\Qtab{0.6cm}{I guess what you mean is that here
%the different answer options are below each other.}}\par
%}
%
%\Qitem[a]{ \Qq{Please describe your first impression.} \Qlines{2} }
%
%\Qitem[b]{ \Qq{In case you would like some more lines to write, here
%they are:} \Qlines{4} }
%
%\section*{Another feature}
%
%\noindent Now, here is another nice feature: you can automatically
%alter the background color of the items. Here is how it can look like.
%Because we are too lazy to think of new questions we% will just ask you
%the same questions again. There is one difference to the above though,
%can you find it?
%
%\renewcommand{\QO}{$\ocircle$}% Use circles now instead of boxes.
%
%\section*{About you}
%
%\QItem{ \Qq{Your name}: \Qline{8cm} }
%
%\QItem{ \Qq{How old are you?} I am \Qline{1.5cm} years old.}
%
%\QItem{ \Qq{Are you in a good mood right now?} \hskip0.4cm \QO{}
%absolutely \hskip0.5cm \QO{} not really because: \Qline{3cm} }
%
%\QItem{ \Qq{Do you like having an option?}
%\begin{Qlist}
%\item Yes, this meets my need for autonomy.
%\item No, I'd rather have someone else decide for me.
%\item Not sure.
%\end{Qlist}
%}
%
%\QItem{ \Qq{What kind of music do you like best?}
%\begin{Qlist}
%\item Johann Sebastian Bach
%\item Jazz
%\item The Beatles
%\item Other: \Qline{4cm}
%\end{Qlist}
%}
%
%\minisec{Please evaluate the following composers}
%\vskip.5em
%
%\QItem[a]{ \Qq{Mozart} \Qtab{3cm}{horrible \Qrating{5} fantastic}}
%
%\QItem[b]{ \Qq{Beethoven} \Qtab{3cm}{horrible \Qrating{5} fantastic}}
%
%\QItem[c]{ \Qq{Johann S. Bach} \Qtab{3cm}{horrible \Qrating{5} fantastic}}
%
%\QItem[d]{ \Qq{John Lennon}\Qtab{2.5cm}{\parbox[t]{3.3cm}{\raggedleft
%Uh, well, I do not like his music very much}
%\Qrating{7} \parbox[t]{3cm}{Oh, yes, you know, really great
%stuff}}}
%
%\QItem[e]{ \Qq{Elvis}\Qtab{2.5cm}{\parbox[t]{3.3cm}{\raggedleft
%Uh, well, I do not like his music very much}
%\Qrating{7} \parbox[t]{3cm}{Oh, yes, you know, really great
%stuff}}}
%
%\section*{About this questionnaire}
%
%\QItem{\Qq{Do you like the style so far?} \Qtab{5.5cm}{\QO{} Yes
%\hskip0.5cm \QO{} No}}
%
%\QItem{\Qq{Is it really worth the ink?} \Qtab{5.5cm}{\QO{} Guess so.
%\hskip0.5cm \QO{} Probably not. \hskip0.5cm \QO{} Don't know.}}
%
%\QItem{\Qq{How does this item look different from the previous one?}
%\Qtab{10.5cm}{\QO{}\Qtab{0.6cm}{Oh. Does it?}}\par
%\Qtab{10.5cm}{\QO{}\Qtab{0.6cm}{No clue. I just can't figure it out.
%So sorry.}}\par
%\Qtab{10.5cm}{\QO{}\Qtab{0.6cm}{I guess what you mean is that here
%the different answer options are below each other.}}\par
%}
%
%\QItem[a]{ \Qq{Please describe your first impression.} \Qlines{2} }
%
%\QItem[b]{ \Qq{In case you would like some more lines to write, here
%they are:} \Qlines{4} }
\hrulefill
\subsection*{Game outcome (to be completed by the researcher)}
\Qitem{ \Qq{Game outcome:} \hskip0.5cm \QO{} The player won \hskip0.9cm \QO{} The robot won \hskip0.9cm \QO{} Abandon}
\Qitem{ \Qq{Time:} \Qline{3cm}}
\Qitem{ \Qq{Notes:} \Qlines{2}}

\section{Italian version}
\subsection*{Chi sei?}

%\Qitem{ \Qq{Nome}: \Qline{8cm} }

\Qitem{ \Qq{Et\`a (anni compiuti):} \Qline{1.5cm} }

\Qitem{ \Qq{Genere:} \hskip0.4cm \QO{} M \hskip0.9cm  \QO{} F}

\Qitem[a]{ \Qq{Hai gi\`a avuto esperienze dirette con dei robot?} \hskip0.4cm \QO{}
No \hskip0.9cm \QO{} S\`i }
\Qitem[b]{ \Qq{Se hai risposto ``S\`i'' alla domanda precedente, racconta la tua esperienza:} \Qlines{4}}
\Qitem{ \Qq{Come giochi di solito?} \newline \QO{} Videogiochi \hskip0.7cm \QO{} Giochi in scatola \hskip0.7cm \QO{} Carte \hskip0.7cm \QO{} Sport \hskip0.7cm \QO{} Altro \Qline{5.1 cm}}
%\Qitem{ \Qq{Do you like having an option?}
%\begin{Qlist}
%\item Yes, this meets my need for autonomy.
%\item No, I'd rather have someone else decide for me.
%\item Not sure.
%\end{Qlist}
%}
%
%\Qitem{ \Qq{What kind of music do you like best?}
%\begin{Qlist}
%\item Johann Sebastian Bach
%\item Jazz
%\item The Beatles
%\item Other: \Qline{4cm}
%\end{Qlist}
%}
%
%\minisec{Please evaluate the following composers}
%\vskip.5em
%
%\Qitem[a]{ \Qq{Mozart} \Qtab{3cm}{horrible \Qrating{5} fantastic}}
%
%\Qitem[b]{ \Qq{Beethoven} \Qtab{3cm}{horrible \Qrating{5} fantastic}}
%
%\Qitem[c]{ \Qq{Johann S. Bach} \Qtab{3cm}{horrible \Qrating{5}
%fantastic}}
%
%\Qitem[d]{ \Qq{John Lennon}\Qtab{2.5cm}{\parbox[t]{3.3cm}{\raggedleft
%Uh, well, I do not like his music very much}
%\Qrating{7} \parbox[t]{3cm}{Oh, yes, you know, really great
%stuff}}}
%
%\Qitem[e]{ \Qq{Elvis}\Qtab{2.5cm}{\parbox[t]{3.3cm}{\raggedleft Uh,
%well, I do not like his music very much}
%\Qrating{7} \parbox[t]{3cm}{Oh, yes, you know, really great
%stuff}}}

\subsection*{Rispetto al gioco appena fatto, indica quanto sei d'accordo con le seguenti affermazioni, dove: 1 = ``per niente d'accordo'', 3 = ``indifferente'' e 5 = ``molto d'accordo''}

\Qitem{ \Qq{Mi sono divertito} \newline \QO{} 1 \hskip0.9cm \QO{} 2 \hskip0.9cm \QO{} 3 \hskip0.9cm \QO{} 4 \hskip0.9cm \QO{} 5}
\Qitem{ \Qq{Avevo voglia di giocare} \newline \QO{} 1 \hskip0.9cm \QO{} 2 \hskip0.9cm \QO{} 3 \hskip0.9cm \QO{} 4 \hskip0.9cm \QO{} 5}
\Qitem{ \Qq{Quando giocavo stavo attento al robot} \newline \QO{} 1 \hskip0.9cm \QO{} 2 \hskip0.9cm \QO{} 3 \hskip0.9cm \QO{} 4 \hskip0.9cm \QO{} 5}
\Qitem{ \Qq{Il robot era attento a quel che facevo} \newline \QO{} 1 \hskip0.9cm \QO{} 2 \hskip0.9cm \QO{} 3 \hskip0.9cm \QO{} 4 \hskip0.9cm \QO{} 5}
\Qitem{ \Qq{Il robot voleva vincere} \newline \QO{} 1 \hskip0.9cm \QO{} 2 \hskip0.9cm \QO{} 3 \hskip0.9cm \QO{} 4 \hskip0.9cm \QO{} 5}
\Qitem{ \Qq{Avevo paura del robot} \newline \QO{} 1 \hskip0.9cm \QO{} 2 \hskip0.9cm \QO{} 3 \hskip0.9cm \QO{} 4 \hskip0.9cm \QO{} 5}
\Qitem{ \Qq{Il gioco dura troppo poco} \newline \QO{} 1 \hskip0.9cm \QO{} 2 \hskip0.9cm \QO{} 3 \hskip0.9cm \QO{} 4 \hskip0.9cm \QO{} 5}
\Qitem{ \Qq{Ho capito subito le regole del gioco} \newline \QO{} 1 \hskip0.9cm \QO{} 2 \hskip0.9cm \QO{} 3 \hskip0.9cm \QO{} 4 \hskip0.9cm \QO{} 5}
\Qitem{ \Qq{Ho cambiato il modo di giocare durante la partita} \newline \QO{} 1 \hskip0.9cm \QO{} 2 \hskip0.9cm \QO{} 3 \hskip0.9cm \QO{} 4 \hskip0.9cm \QO{} 5}
\Qitem{ \Qq{Il robot ha fatto delle finte} \newline \QO{} 1 \hskip0.9cm \QO{} 2 \hskip0.9cm \QO{} 3 \hskip0.9cm \QO{} 4 \hskip0.9cm \QO{} 5}
\Qitem{ \Qq{Mi divertirei di più se il robot facesse delle finte} \newline \QO{} 1 \hskip0.9cm \QO{} 2 \hskip0.9cm \QO{} 3 \hskip0.9cm \QO{} 4 \hskip0.9cm \QO{} 5}
\Qitem{ \Qq{Era facile capire cosa avrebbe fatto il robot} \newline \QO{} 1 \hskip0.9cm \QO{} 2 \hskip0.9cm \QO{} 3 \hskip0.9cm \QO{} 4 \hskip0.9cm \QO{} 5}

%
%\Qitem{\Qq{Do you like the style so far?} \Qtab{5.5cm}{\QO{} Yes
%\hskip0.5cm \QO{} No}}
%
%\Qitem{\Qq{Is it really worth the ink?} \Qtab{5.5cm}{\QO{} Guess so.
%\hskip0.5cm \QO{} Probably not. \hskip0.5cm \QO{} Don't know.}}
%
%\Qitem{\Qq{How does this item look different from the previous one?}
%\Qtab{10.5cm}{\QO{}\Qtab{0.6cm}{Oh. Does it?}}\par
%\Qtab{10.5cm}{\QO{}\Qtab{0.6cm}{No clue. I just can't figure it out.
%So sorry.}}\par
%\Qtab{10.5cm}{\QO{}\Qtab{0.6cm}{I guess what you mean is that here
%the different answer options are below each other.}}\par
%}
%
%\Qitem[a]{ \Qq{Please describe your first impression.} \Qlines{2} }
%
%\Qitem[b]{ \Qq{In case you would like some more lines to write, here
%they are:} \Qlines{4} }
%
%\section*{Another feature}
%
%\noindent Now, here is another nice feature: you can automatically
%alter the background color of the items. Here is how it can look like.
%Because we are too lazy to think of new questions we% will just ask you
%the same questions again. There is one difference to the above though,
%can you find it?
%
%\renewcommand{\QO}{$\ocircle$}% Use circles now instead of boxes.
%
%\section*{About you}
%
%\QItem{ \Qq{Your name}: \Qline{8cm} }
%
%\QItem{ \Qq{How old are you?} I am \Qline{1.5cm} years old.}
%
%\QItem{ \Qq{Are you in a good mood right now?} \hskip0.4cm \QO{}
%absolutely \hskip0.5cm \QO{} not really because: \Qline{3cm} }
%
%\QItem{ \Qq{Do you like having an option?}
%\begin{Qlist}
%\item Yes, this meets my need for autonomy.
%\item No, I'd rather have someone else decide for me.
%\item Not sure.
%\end{Qlist}
%}
%
%\QItem{ \Qq{What kind of music do you like best?}
%\begin{Qlist}
%\item Johann Sebastian Bach
%\item Jazz
%\item The Beatles
%\item Other: \Qline{4cm}
%\end{Qlist}
%}
%
%\minisec{Please evaluate the following composers}
%\vskip.5em
%
%\QItem[a]{ \Qq{Mozart} \Qtab{3cm}{horrible \Qrating{5} fantastic}}
%
%\QItem[b]{ \Qq{Beethoven} \Qtab{3cm}{horrible \Qrating{5} fantastic}}
%
%\QItem[c]{ \Qq{Johann S. Bach} \Qtab{3cm}{horrible \Qrating{5} fantastic}}
%
%\QItem[d]{ \Qq{John Lennon}\Qtab{2.5cm}{\parbox[t]{3.3cm}{\raggedleft
%Uh, well, I do not like his music very much}
%\Qrating{7} \parbox[t]{3cm}{Oh, yes, you know, really great
%stuff}}}
%
%\QItem[e]{ \Qq{Elvis}\Qtab{2.5cm}{\parbox[t]{3.3cm}{\raggedleft
%Uh, well, I do not like his music very much}
%\Qrating{7} \parbox[t]{3cm}{Oh, yes, you know, really great
%stuff}}}
%
%\section*{About this questionnaire}
%
%\QItem{\Qq{Do you like the style so far?} \Qtab{5.5cm}{\QO{} Yes
%\hskip0.5cm \QO{} No}}
%
%\QItem{\Qq{Is it really worth the ink?} \Qtab{5.5cm}{\QO{} Guess so.
%\hskip0.5cm \QO{} Probably not. \hskip0.5cm \QO{} Don't know.}}
%
%\QItem{\Qq{How does this item look different from the previous one?}
%\Qtab{10.5cm}{\QO{}\Qtab{0.6cm}{Oh. Does it?}}\par
%\Qtab{10.5cm}{\QO{}\Qtab{0.6cm}{No clue. I just can't figure it out.
%So sorry.}}\par
%\Qtab{10.5cm}{\QO{}\Qtab{0.6cm}{I guess what you mean is that here
%the different answer options are below each other.}}\par
%}
%
%\QItem[a]{ \Qq{Please describe your first impression.} \Qlines{2} }
%
%\QItem[b]{ \Qq{In case you would like some more lines to write, here
%they are:} \Qlines{4} }
\hrulefill
\subsection*{Risultato gioco (a cura del ricercatore)}
\Qitem{ \Qq{Esito partita:} \hskip0.5cm \QO{} Ha vinto il giocatore \hskip0.9cm \QO{} Ha vinto il robot \hskip0.9cm \QO{} Abbandono}
\Qitem{ \Qq{Tempo:} \Qline{3cm}}
\Qitem{ \Qq{Note:} \Qlines{2}}