\chapter*{Abstract}
% EWERTON: How long should the abstract be? Short, no?

As our technology grows new game experiences emerges. Among those appears a new type of game, where human players are involved in a often quite demanding physical activity against robotic agents. This type of games has been introduced as~\gls{pirg}. In this thesis work, we have developed methods and insights for modeling players in a~\gls{pirg} environment with data from on-board sensors processed in real-time.

This new type of game environment has as main characteristic the exploitation of the real world (in both its dynamical, unstructured, and structured aspects) as environment, and of one or more real, physical, autonomous robots as game opponents or companions. The ultimate direction for~\gls{pirg} is to obtain a robotic player aiming purposefully maximizing human player entertainment. In our work, we provide a panorama of design for such robotic applications, advocating, in the process, the benefits of~\gls{ml} techniques in order to tackle the challengers. We present methods and insights for player modeling using a subset of such techniques as well as direction for future research towards the achieving adaptation.

Besides being interesting field for testing approaches from~\glsdesc{ml}, a~\gls{pirg} scenario is a challenging study environment for several other disciplines, like:~\gls{ai} (general and specific); Statistics;~\gls{hri}; Robotics; Psychology; Design, among others.



