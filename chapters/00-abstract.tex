\chapter*{Abstract}
% EWERTON: How long should the abstract be? Short, no?

As technology progresses new game experiences emerges. Among these, a new type of game appears, where human players are involved in a physical activity against robotic agents. This type of games has been introduced as~\gls{pirg}. In this work, we have developed methods and insights for modeling players in a~\gls{pirg} environment with data from on-board sensors processed in real-time.

This new type of game environment has as main characteristic the exploitation of the real world as environment (in both its dynamical, unstructured, and structured aspects), and of one or more real, physical, autonomous robots as game opponents or companions. The ultimate direction for~\gls{pirg} is to obtain a robotic player purposefully aiming at maximizing human player entertainment. In our work, we provide a panorama of design for such robotic applications, advocating, in the process, the benefits of~\gls{ml} techniques to tackle the challenges. We present methods and insights for player modeling using \gls{ml} techniques, as well as direction for future research to achieve full adaptation.

Besides being an interesting field for testing approaches from~\glsdesc{ml}, a~\gls{pirg} scenario provides a challenging application for several other disciplines, among which: (general and specific)~\gls{ai}, Statistics,~\gls{hri}, Robotics, Psychology, Design.



