\chapter{Introduction}
%\epigraph{\itshape Virtute enim ipsa non tam multi praediti esse quam videri volunt}{--- Cicero, De Amicitia, Cap. XCVIII}

What seems to be a natural evolution for game playing experience %TODO this is really too abrupt. First, the majority of game experiences is done without video, and will always be done in the physical world with or without objects ("toys"), second a first sentence introducing the topic of the thesis would be more appropriate for an introduction.
is to bring the elimination of screens and devices in order to present the users with the possibility to physically interact with autonomous agents in their homes without the need to produce a virtual reality. This pretty new style of games has been recently defined as~\gls{pirg} and has as its main objective the exploitation of the real world (in both its dynamical, unstructured, and structured aspects) as environment, and of one or more real, physical, autonomous robots as game opponents or companions~\citep{martinoia_physically_2013}.

Like commercial virtual games, the main aspect of~\gls{pirg}s is to produce a sense of entertainment and pleasure that can be ``consumed'' by a large number of users\footnote{In this work, since the player is an user for the gaming application both words ``player(s)'' and ``user(s)'' will be used interchangeably.}. Furthermore, an important aspect of autonomous robots and systems during the game should be, as expected, an exhibition of rational behavior and, in this sense, they must be capable enough to play the role of opponents or teammates effectively, since by practical means people tend to avoid to play with or against a dull entity~\citep{martinoia_physically_2013}.
To come up with appropriate robotic players, it is possible to extract knowledge from the behavior of co-players and to implement some mechanism for modeling them, in particular their preference for specific actions and interaction patterns. %TODO be careful here, since your thesis is not producing this. EWERTON: But this is a general statement about what is possible. I don't see the problem.
Being able to recognize co-player's intention may substantially improve the capacity of taking proper decisions about the actions to take. At least in the human's perspective, this ability is critical since interpersonal interaction presupposes understanding  motivations and high-level plans, as well as forecast of future events~\citep{sukthankar_plan_2014}.

When it comes to extract useful information from agent's behavior, one can see at least two related, approaches: \begin{inparaenum}[\itshape a\upshape)]\item Modeling for competitive advantage and \item Modeling for experience optimization\end{inparaenum}. In the former, techniques are oriented towards obtaining competitive advantage through interaction patterns. %TODO I cannot catch what you would mean with the previous sentence EWERTON: Changed most of the paragraph.
Against an adversary with private strategies and conflicting goals it is necessary to adapt to the dynamics of the situation caused by the game play~\citep{rofer_overview_2012}. In essence, this means that it is vital to pay attention to any information from the opponent's behavior that might help to optimize the decision making process and find appropriate actions.

The focus on modeling behavior for experience optimization, however, is much related to the idea of extracting useful features from users in order to adjust parameters that are correlated with their experience in the activity, for the sake of offering a better product or helping the user to achieve some particular goals. In a game scenario, this notion is commonly applied when designers attempt to define a mechanism capable of adjusting the difficulty or general appearance of the game in the expectation of rising the player's entertainment. 

Very traditionally, the sense of game difficulty is designed to increase along the course of the experience, and it can either happen in a linear fashion or through steps represented by the levels or phases, where a player is forced to select the difficulty level through a set of discrete options (easy, medium, hard, very hard). However, very often this ``static'' way of setting up a difficulty curve is not accurate enough and it may not account for the difference between players or even their different learning rates. %TODO Here the consequence is missing "given ...<very long premise>.' One would expect that given that something comes. EWERTON: Removed the given.
In principle, it turns out useful and natural to think about coming up with modeling techniques that may empower the play experience. 

In summary, it would be important to have an autonomous robot playing in a~\gls{pirg} and able to automatically adjust its behavior such that it may likely match the user's skill and, by doing so, maintain the user engaged and entertained. Also, it is important to make this robot appear rational, or, possibly, smart enough for the specific player. 

In this thesis work, after providing a panorama of approaches that take inspiration from~\gls{ai} and~\gls{ml} techniques in order to tackle the problem of modeling co-operating agent's behavior and activity in games and robots, we propose models and report results showing effort in addressing the design of better %TODO Comparatives always need a term to be compared with
agents for~\gls{pirg}.%TODO as a goal of a PhD thesis this is really understated. I would put it as a first sentence of the introduction (and then repeat it here), and would say that you have developed methods to model the player in a PIRG with data from on-board sensors, in real-time, to obtain a robotic player aiming at maximizing engagement (or entertainment, this is still to be defined) of the human player.

The scope of this document is  focused on model proposals that can latently model player activities and general behavior. %TODO if this is the (a posteriori ;-) ) goal of the research, it has to be included in the first sentence as the one mentioned just above. The scope is NOT the scope of THIS DOCUMENT, but the scope of your RESEARCH, described in this document.

\section{Research questions, hypothesis and objectives}\label{sec:research_question}
Since in any kind of~\gls{pirg} autonomous robots are supposed to be perceived as smart entities, the key point for investigation deals with finding good answers for the following questions:

\begin{itemize}
\item How to discover player types and characterize player behavior in~\gls{pirg}? %TODO can a behavior be quantified? What does it mean? Maybe "model" or "characterize"? EWERTON: Changed quantify into characterize
\item How is it possible to adapt the behavior of the robot to optimize the satisfaction of the human player?
\item To what extent does adaptation impact reported entertainment? %TODO Do we have any answer to this question? Don't we miss at least the baseline? EWERTON: mmm... those items are supposed to be general aspects for PIRG design. Our objectives are the next itemize below. Anyway, our work on deception implicitly touches on this.
\end{itemize}

From this, the objective of this thesis focuses on the exploitation of~\gls{ml} techniques to design~\gls{pirg} robotic agents, that could engage playmates by possibly performing behavior/strategy adjustment. The hypothesis is that~\gls{ml} could make it possible to implement adaptation so to decrease appropriately the predictability of robot behavior and introduce game dynamics that could empower player’s engagement, making the robot well accepted as game companion.

We emphasize that, in the process, a special attention has to be given to the development of the robot's behavior so that it is not likely to be interpreted by the player as a random one. In other words, when playing, the robot must show a level of rationality that matches the expectation of the players, making them able to foresee its goals, although the selected actions have a predictability appropriate for the specific player. Rationality here consists in selecting actions upon the evaluation of world states %TODO Which world states? Are you evaluating single world states? EWERT: General world states. The concept of rationality is bounded to them in the sense that it emerges from selecting the action leading to a better world state (see Russel and Norvig).
by a \textit{performance measure}, also called \textit{utility function}~\citep{russell_artificial_2009}. In the~\gls{pirg} context, whatever performance measure is used it should drive the robot behavior towards well defined and interpretable actions, but yet not so that the robot attitude becomes predictable. Predictability is reported to be a negative impact in the quality of engagement.   

Specifically, the general goals of this thesis were:

\begin{itemize}
\item To investigate and report aspects of~\gls{pirg} design with the aim of constructing an enjoyable game interaction; 
\item To understand and report phenomena that impact player engagement through the use of a mobile adversarial robot in a~\gls{pirg} environment;
\item To investigate relevant concepts of design of mobile agents for~\gls{pirg};
\item To implement user's behavior modeling in~\gls{pirg} robots by reasoning about data coming from player tracking in the shortest possible time, as required by the~\gls{pirg} setting;
\item To explore ways of robot behavior adjustment using information about past interactions (player's typical behaviors, preferences, etc.).
\end{itemize}

It is assumed that autonomous and learning systems that encompass perception, action, and communication in a unified and principled way via~\gls{ml}-based techniques lay at the core of a new frontier for robotics, and~\gls{pirg}s in particular. During our investigation we also aimed at keeping control on technical constraints in order to enable the spread of~\gls{pirg} in the society, making them possibly reach the large scale market. In particular, we explored the use of cheap sensors and algorithms requiring little power (``green algorithms'') to be executed in real time and operating in non-structured environments, 

\section{Thesis outline}
The thesis is structured in the following chapters.

\begin{itemize}
\item\emph{Chapter~\ref{ch:art}} provides an introduction to physically interactive games with robotic agents that have been proposed in literature, and presents some design guidelines. %TODO Maybe it would be worth also finding a way  to mention the work of our POLITO reviewer, who implemented so-called "phygames". MAybe just a citation: they have published a couple of papers... ;-)
\item\emph{Chapter~\ref{ch:foundation}:} The game environment and robot platform are at the core of a~\gls{pirg} application. In this chapter, we detail the designed environment and adopted robotic platform.
\item\emph{Chapter~\ref{ch:playing_for_advantage}:} will provide a brief overview of popular approaches to design artificial systems able to play for competitive advantage, and we present some of our concerns about this. 
\item\emph{Chapter~\ref{ch:review_playing_optimization}:} gives a structured overview of literature on player modeling as well as details regarding the position of our work on it.
\item\emph{Chapter~\ref{ch:activity}:} One step towards effective player modeling is the implementation of an activity recognition system. In this chapter we describe the efforts on exploiting a simple input data transformation for the recognition of motion primitives in acceleration patterns akin to archetypal activities in the game scenario.
\item\emph{Chapter~\ref{ch:modeling}:} Player behavior modeling is the backbone of adaptive behavior strategy for playing robots. The chapter presents a new proposal for latent player behavior modeling.
\item\emph{Chapter~\ref{ch:deception}:} Engagement is believed to be related to several factors and one of such is the level of information about the opponent's actions. In this chapter, we present a study case on the use of deceptive motion during play.
\item\emph{Chapter~\ref{ch:adaptation}:} Adaption is by no means a trivial task specially in the context of~\gls{pirg}. In this chapter, we detail insights of a system to actively select game parameters appropriate for the specific human player.
%TODO What about the last one? EWERTON: What do you mean by the last one?
\item\emph{Chapter~\ref{ch:future}:} Concludes the thesis and details further directions for our research.
\end{itemize}

\section{Paper contributions}

This thesis is partially based on expanded versions of the following publications:

\begin{itemize}
    \item Ewerton Oliveira, Luca Morreale, Davide Orrù, Tiago Nascimento and Andrea Bonarini. Learning and mining player motion profiles in Physically Interactive Robogames. Future Internet, 10(3), 2018. ISSN 1999-5903. doi:10.3390/fi10030022\\ URL~\url{http://www.mdpi.com/1999-5903/10/3/22};
    \item Ewerton Oliveira, Davide Orrù, Tiago Nascimento and Andrea Bonarini. Modeling Player Activity  in a Physical Interactive Robot Game Scenario. In Proceedings of the 5th International Conference on Human Agent Interaction, HAI'17. pages 411-414, New York, NY, USA, 2017b. ACM. ISBN: 978-1-4503-5113-3. \\
    doi:10.1145/3125739.3132608 URL \url{http://doi.acm.org/10.1145/3125739.3132608};
    \item Ewerton Oliveira, Davide Orrù, Tiago Nascimento and Andrea Bonarini. Activity Recognition in a Physical Interactive RoboGame. In 2017 Joint IEEE International Conference on Development and Learning and Epigenetic Robotics, ICDL-EpiRob 2017, Lisbon, Portugal, September 18-21 2017, pages 1-6, Lisbon, 2017.
\end{itemize}

Papers in progress:

\begin{itemize}
\item Adjusting robot playing behavior through latent skill modeling for increasing player satisfaction;
\item Key issues in designing engaging physically interactive robogames: a developmental account;
\item Using social interaction analysis and deception for entertainment support in a physically interactive robogame.
\end{itemize}