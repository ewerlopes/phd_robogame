\chapter{Introduction}
%\epigraph{\itshape Virtute enim ipsa non tam multi praediti esse quam videri volunt}{--- Cicero, De Amicitia, Cap. XCVIII}

In this thesis work, we have developed methods and insights for modeling players in a~\acrfull{pirg} with data from on-board sensors, in real-time, to obtain a robotic player aiming at the ultimate goal of maximizing human player entertainment. We begin by providing a panorama of design for such robotic applications, advocating, in the process, the benefits of~\gls{ml} techniques in order to tackle the challengers.

A~\gls{pirg} is a new type of game environment whose main characteristic is the exploitation of the real world (in both its dynamical, unstructured, and structured aspects) as environment, and of one or more real, physical, autonomous robots as game opponents or companions~\citep{martinoia_physically_2013}.

Like commercial virtual games, the main aspect of~\gls{pirg}s is to produce a sense of entertainment and pleasure that can be ``consumed'' by a large number of users\footnote{In this work, since the player is an user for the gaming application both words ``player(s)'' and ``user(s)'' will be used interchangeably.}. Furthermore, an important aspect of autonomous robots and systems during the game should be, as expected, the exhibition of rational behavior and, in this sense, they must be capable enough to play the role of opponents or teammates effectively, since by practical means people tend to avoid to play with or against a dull entity~\citep{martinoia_physically_2013}.

In the process of coming up with appropriate robotic players, it might be possible to extract knowledge from the behavior of co-players and to implement some mechanism for modeling them, in particular their preference for specific actions and interaction patterns. %TODO be careful here, since your thesis is not producing this. EWERTON: But this is a general statement about what is possible. I don't see the problem. ANDY You are in the introduction and you are introducing the problem that you faced. Proposing a different problem then rises expectations about your work, and the reviewer may complain about missing parts of the research. If you introduce what is the relevance of modeling the opponent in general, then one may take what you have done as a valid contribution in a wider context and will not perceive the thesis as a promise non satisfied. If you want to enter in the specific, please enter in the specific issues you have faced, to mention their relevance and contribution to solve issues. In   particular, you may talk about user attitude, user engagement, user activity, possibly not user intentions.
Being able to recognize co-player's intention may substantially improve the capacity of taking proper decisions about the actions to take. At least in the human's perspective, this ability is critical since interpersonal interaction presupposes understanding motivations and high-level plans, as well as forecast of future events~\citep{sukthankar_plan_2014}.

When it comes to extract useful information from agent's behavior, one can see at least two related, approaches: \begin{inparaenum}[\itshape a\upshape)]\item Modeling for competitive advantage and \item Modeling for experience optimization\end{inparaenum}. In the former, techniques are oriented towards obtaining competitive advantage through interaction patterns without regarding the entertainment level of co-players directly. 
Against an opponent with private strategies and conflicting goals it is necessary to adapt to the dynamics of the situation caused by the game play~\citep{rofer_overview_2012}. In essence, it is vital to pay attention to any information from the behavior of opponent's and co-players that might help to optimize the decision making process and find appropriate actions.

The second approach, which focuses on modeling behavior for experience optimization, is more focused on adjusting parameters that are correlated with their experience in the activity, for the sake of offering a better product or helping the user to achieve some particular goals. In a game scenario, this notion is commonly applied when designers attempt to define a mechanism capable of adjusting the difficulty or general appearance of the game in the expectation of rising the player's entertainment.

In our research, these approaches complement each other towards the overall objective of designing entertaining~\glspl{pirg}. One direction for achieving this objective is the adaptation of the behavior of agents towards regulating the difficulty perceived by the players. Very traditionally, the sense of game difficulty is designed to increase along the course of the experience, and it can either happen in a linear fashion or through steps represented by the levels or phases, where a player is forced to select the difficulty level through a set of discrete options (easy, medium, hard, very hard). Unfortunately, this ``static'' way of setting up a difficulty curve is often not accurate, since it may not account for the difference between players or even their different learning rates. 

In~\glspl{pirg}, the experience is bounded to fundamental aspects of the platform, \eg cost and safety, and the interaction often take place in a real-world scenario without being structured in levels (with different morphology and elements) of increasing difficulty. The object of adaptation is the agent's ability to play. Coming up with modeling techniques to keep track of co-players and empower such ability is required.

In summary, it would be important to have an autonomous playing robot able to automatically adjust its behavior such that it may likely match the user's ability to play (\textit{skill}), and, by doing so, maintain the user engaged and entertained. In this process, it is important to make this robot appear rational, or, possibly, smart enough for the specific player. Rationality in this context helps to ease the interpretation of actions and makes the agent believable as a true opponent, impacting the willingness to play~\citep{martinoia_physically_2013, bonarini_timing_2014}. 

We provide ways for characterizing player behavior towards ultimately allowing adaptation to take place. Our contributions vary from the proposal of a method for real-time activity recognition to the use of a popular algorithm from the Data Mining community, called~\acrfull{lda}, for the purpose of categorizing player behavior. Besides showing the applicability of~\gls{lda} to our~\gls{pirg} scenario, we further demonstrate how it can be used in a the continuous data setup.

Additionally, we studied the use of deception to support entertainment and report the appreciation of the game by the players, discussing the importance of having a well designed deceptive mechanism. Our work also brings further insights on how one can use~\gls{lda} and~\gls{pmf} to frame the problem of adapting the a robot behavior using a~\gls{cf} approach -- a popular technique for recommendation systems.

% after providing a panorama of approaches that take inspiration from~\gls{ai} and~\gls{ml} techniques in order to tackle the problem of modeling co-operating agent's behavior and activity in games and robots, we propose models and report results showing effort in addressing the design of better %TODO Comparatives always need a term to be compared with. EWERTON: Comparative of what? ANDY: better is a comparative 
% agents for~\gls{pirg}. %TODO as a goal of a PhD thesis this is really understated. I would put it as a first sentence of the introduction (and then repeat it here), and would say that you have developed methods to model the player in a PIRG with data from on-board sensors, in real-time, to obtain a robotic player aiming at maximizing engagement (or entertainment, this is still to be defined) of the human player. 
% The scope of this document is  focused on model proposals that can latently model player activities and general behavior. %TODO if this is the (a posteriori ;-) ) goal of the research, it has to be included in the first sentence as the one mentioned just above. The scope is NOT the scope of THIS DOCUMENT, but the scope of your RESEARCH, described in this document.

\section{Research questions, hypothesis and objectives}\label{sec:research_question}
In any kind of~\gls{pirg}, autonomous robots are supposed to be perceived as smart entities~\citep{martinoia_physically_2013, bonarini_timing_2014}. There are some global points for investigation in this area including:

\begin{itemize}
\item How to discover player types and characterize player behavior in~\gls{pirg}? 
\item How is it possible to adapt the behavior of the robot to optimize the satisfaction of the human player?
\item To what extent does adaptation impact reported entertainment? %TODO Do we have any answer to this question? Don't we miss at least the baseline? EWERTON: mmm... those items are supposed to be general aspects for PIRG design. Our objectives are the next itemize block below. Anyway, our work on deception implicitly touches on this. ANDY The same motivations I have introduced above call for mentioning only the aspects you worked on or general issues. Let's avoid to mention detailed issues you did not really face. We have not been able to evaluate "to what extent", not yet.
\end{itemize}

From those general aspects, the objective of this thesis focuses on the exploitation of~\gls{ml} techniques to design~\gls{pirg} robotic agents, that could engage playmates by possibly performing behavior/strategy adjustment. The hypothesis is that~\gls{ml} could make it possible to implement adaptation so to decrease appropriately the predictability of robot behavior and introduce game dynamics that could empower player’s engagement, making the robot well accepted as game companion.

We emphasize that, in the process, a special attention has to be given to the development of the robot's behavior so that it is not likely to be interpreted by the player as a random one. In other words, when playing, the robot must show a level of rationality that matches the expectation of the players, making them able to foresee its goals, although the selected actions have a predictability appropriate for the specific player. Rationality here consists in selecting actions upon the evaluation of world states by a \textit{performance measure}, also called \textit{utility function}~\citep{russell_artificial_2009}. In the~\gls{pirg} context, whatever performance measure is used it should drive the robot behavior towards well defined and interpretable actions, but yet not so that the robot attitude becomes predictable. Predictability is reported to be a negative impact in the quality of engagement.   

Specifically, the general goals of this thesis were:

\begin{itemize}
\item To investigate and report aspects of~\gls{pirg} design with the aim of constructing an enjoyable game interaction; 
\item To understand and report phenomena that impact player engagement through the use of a mobile adversarial robot in a~\gls{pirg} environment;
\item To investigate relevant concepts of design of mobile agents for~\gls{pirg};
\item To implement user's behavior modeling in~\gls{pirg} robots by reasoning about data coming from player tracking in the shortest possible time, as required by the~\gls{pirg} setting;
\item To explore ways of robot behavior adjustment using information about past interactions (player's typical behaviors, preferences, etc.).
\end{itemize}

It is assumed that autonomous and learning systems that encompass perception, action, and communication in a unified and principled way via~\gls{ml}-based techniques lay at the core of a new frontier for robotics, and~\gls{pirg}s in particular. During our investigation we also aimed at keeping control on technical constraints in order to enable the spread of~\gls{pirg} in the society, making them possibly reach the large scale market. In particular, we explored the use of cheap sensors and algorithms requiring little power (``green algorithms'') to be executed in real time and operating in non-structured environments, 

\section{Thesis outline}
The thesis is structured in the following chapters.

\begin{itemize}
\item\emph{Chapter~\ref{ch:art}} provides an introduction to physically interactive games with robotic agents that have been proposed in literature, and presents some design guidelines. %TODO Maybe it would be worth also finding a way  to mention the work of our POLITO reviewer, who implemented so-called "phygames". Maybe just a citation: they have published a couple of papers... ;-)
% EWERTON: I will do that soon. Please do not erase this comment. 
\item\emph{Chapter~\ref{ch:foundation}:} The game environment and robot platform are at the core of a~\gls{pirg} application. In this chapter, we detail the designed environment and adopted robotic platform.
\item\emph{Chapter~\ref{ch:playing_for_advantage}:} will provide a brief overview of popular approaches to design artificial systems able to play for competitive advantage, and we present some of our concerns about this. 
\item\emph{Chapter~\ref{ch:review_playing_optimization}:} gives a structured overview of literature on player modeling as well as details regarding the position of our work on it.
\item\emph{Chapter~\ref{ch:activity}:} One step towards effective player modeling is the implementation of an activity recognition system. In this chapter we describe the efforts on exploiting a simple input data transformation for the recognition of motion primitives in acceleration patterns akin to archetypal activities in the game scenario.
\item\emph{Chapter~\ref{ch:modeling}:} Player behavior modeling is the backbone of adaptive behavior strategy for playing robots. The chapter presents a new proposal for latent player behavior modeling.
\item\emph{Chapter~\ref{ch:deception}:} Engagement is believed to be related to several factors and one of such is the level of information about the opponent's actions. In this chapter, we present a study case on the use of deceptive motion during play.
\item\emph{Chapter~\ref{ch:adaptation}:} Adaption is by no means a trivial task specially in the context of~\gls{pirg}. In this chapter, we detail insights of a system to actively select game parameters appropriate for the specific human player.
\item\emph{Chapter~\ref{ch:future}:} Concludes the thesis and details further directions for our research.
\end{itemize}

\section{Paper contributions}

This thesis is partially based on expanded versions of the following publications:

\begin{itemize}
    \item Ewerton Oliveira, Luca Morreale, Davide Orrù, Tiago Nascimento and Andrea Bonarini. Learning and mining player motion profiles in Physically Interactive Robogames. Future Internet, 10(3), 2018. ISSN 1999-5903. doi:10.3390/fi10030022\\ URL~\url{http://www.mdpi.com/1999-5903/10/3/22};
    \item Ewerton Oliveira, Davide Orrù, Tiago Nascimento and Andrea Bonarini. Modeling Player Activity  in a Physical Interactive Robot Game Scenario. In Proceedings of the 5th International Conference on Human Agent Interaction, HAI'17. pages 411-414, New York, NY, USA, 2017b. ACM. ISBN: 978-1-4503-5113-3. \\
    doi:10.1145/3125739.3132608 URL \url{http://doi.acm.org/10.1145/3125739.3132608};
    \item Ewerton Oliveira, Davide Orrù, Tiago Nascimento and Andrea Bonarini. Activity Recognition in a Physical Interactive RoboGame. In 2017 Joint IEEE International Conference on Development and Learning and Epigenetic Robotics, ICDL-EpiRob 2017, Lisbon, Portugal, September 18-21 2017, pages 1-6, Lisbon, 2017.
\end{itemize}

Papers in progress:

\begin{itemize}
\item Adjusting robot playing behavior through latent skill modeling for increasing player satisfaction;
\item Key issues in designing engaging physically interactive robogames: a developmental account;
\item Using social interaction analysis and deception for entertainment support in a physically interactive robogame.
\end{itemize}