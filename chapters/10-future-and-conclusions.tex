\chapter{Future work and conclusion}\label{ch:future}

There is a series of future developments for the work we have conducted. We begin by the futures works related to the~\gls{lda} based approach in chapter~\ref{ch:modeling}.

\section{Future research directions}
\subsection{Latent modeling}\label{sec:future_lda}
As future work, one alternative could be the incorporation of \textit{proximity} information in order to capture how distance patterns relate to playing style and interest. This measure, similar to what has been done in section~\ref{sec:simple_model}, would add up to the physical activity description and allow for a fine-grained topic description.

Another opportunity for improvements is the incorporation of time dependency between the extracted windows. In other words, one would go beyond the \textit{bag-of-words} representation explored by the basic~\gls{lda}. We believe the incorporation of time dependencies would help to describe variations in playing style during play, given the opportunity to detect change points in styles and favor game personalization through robot behavior adaptation. This also may help to avoid the~\gls{rbe}, which is defined as the oscillatory system instability often present when designing auto-adjusting game systems and translates to making the game too easy when too hard, and too hard when too easy. This oscillation is a concern since the player would perceive the game as trying to explicitly adjust to his behavior and, thus, have a large impact on engagement~\citep{martinoia_physically_2013}.

Vector quantization is a standard practice and it is used across different applications,~\eg acoustic topic models~\citep{kim_acoustic_2009,kim_audio_2009} and scene understanding~\citep{cao_spatially_2007,li_towards_2009,niu_context_2012}. Here, it is necessary for the definition of the multinomial distribution of words that defines each topic. Despite all, vector quantization for obtaining clusters as word-like units has some drawbacks. Among these are the loss of information and the difficulty in clustering the quantization results coherently, mostly in presence of closely spaced elements in the vector space~\citep{hu_latent_2012}.
We believe that it would be interesting to remain in the continuous domain defined by the feature data, instead of relying on a fixed predefined set of tokens, i.e., the vocabulary set used for training the~\gls{lda}. This open new ways to experiment with the model definition.

One important aspect not considered by our model is the role of skill. For example, a highly skilled player could try to minimize his/her energy expenditure by performing precise movements towards winning the game and in theory could perfectly show a low-movement profile.  The incorporation of such notion can occur by imposing new variables to describe the skill and dependencies between them. 

To the best of our knowledge, our work is the first one to model the motion behavior of the player using raw accelerometer signal in conjunction with~\gls{lda} in a~\gls{pirg} scenario. We have demonstrated that the proposed approach is able to cluster and represent different play types learned from data. Our results indicate the approach is able to approximate well human similarity and is believed to be useful to support player modeling in such situation. We expect that further research in the direction of what has been exposed above will provide a definitive proof of the method and better quantify its efficacy. 

\subsection{Deception for entertainment support}
We have implemented a lively, enjoyable, physical robogame, where humans reported the robot as a rational agent, whose goal is to win. From the responses to the post-game questionnaire, deception is an expected component for the robot behavior. It is interesting that such reports are given to a scenario where the robot capabilities are strongly reduced w.r.t. its possible top performance (\eg velocities are reduced due to the need for satisfying safety issues as detailed in chapter~\ref{ch:playing_for_advantage}), yet it is able to provide good capacity to provide fun. In the overall picture, when using deception our robot matches the expectation of interacting people to attribute rationality to the robot companion, which is one of the aims of human-robot interaction.

In the future, it would be interesting to isolate study variables even more and perform a more controlled experiment to estimate the causal relationship with engagement. One supposition, for example, is that there exist a natural bias for players interacting with a robot for the first time in a game situation. We understand that the novelty of the situation tend to positively favor attributes in the robot behavior, such as the perception of rationality. Also, an important direction would be aimed at understanding how to limit factors from the evaluation of engagement itself, so as to effectively estimate the contribution of proposed methods for engagement support in ~\gls{pirg}s. In other words, this could address the Hawthorne Effect~\citep{jones_was_1992} and its implication in the scenario.

The heavy cognitive load required by our game, due to the demand in memory, spatial reasoning and  physical activity, is also suspected to add bias to the overall perception of the game. When the player is busy, some robot misbehavior's may not be taken into account properly, which thus reduces the human player chances of spotting platform defects and lowers the general perception of opponent rationality. Additionally, as reported in our experiments, players often cannot distinguish between the robot's obstacle avoidance movements and those properly aimed at providing deception. A further study on this distinction is necessary in order to understand to what extent deceptive movements are really interpreted as so.

A proper definition of the population of players is also an interesting direction of research. The stratification of groups by age, considering a fine grained separation, would provide useful answers to the question of interpreting the impact of deception in the engagement of players.

Finally, the use of a player model using~\gls{ml} techniques for triggering the deceptive movement would provide means to ease the ability of the player to distinguish the robot's actions. For example, instead of considering only momentary position of the player as the main factor to trigger deception, the system may consider the probability that the player is paying attention to its current movement or, in other words, the system would monitor the probability that the player will interpret the next trajectory as a deceptive one. Therefore, the model would allow deception only when the probability is above a certain predefined value. This, in principle, would help to separate actions well, boost human perception of the game opponent as well as facilitate evaluation of methods.

\subsection{Behavior adaption}
Despite providing theoretical insights to behavior adaption, experimental results are needed for accessing the model capabilities. An immediate direction for future research is thus to implement the~\gls{cer} model in our robot and design experiments to evaluate its impact in the interaction. 

The model in its basic form does not offer conditions to detect changes in the player effort profile in a way filtering algorithms do, for instance. Therefore, a potential issue might be its usage as a real-time adaptation procedure. Nonetheless, we note that an estimation of effort and progress is not needed in a very small scale since it would be prone to~\glsdesc{rbe}, as mentioned above.

As a potential solution, one may define ``checkpoints'' (intended as moments in time) during the game from which to allow the re-estimation of the model parameters and effectively select a new~\gls{ds}. The time distance between the checkpoints will need to be large enough to account for the collection of sufficient information about the player.

Remove some hyper-parameters and eliminate the necessity for model selection is an interesting direction for research as well. In~\gls{cer}, since it is based on a~\gls{lda} model requiring the number of effort topics as hyper-parameters, one possibility is to use a non-parametric approach and select the number of topics from data. There has been several works on the use of Dirichlet Priors and Bayesian non-parametrics for such purpose. A popular work in this matter is~\cite{teh_sharing_2005}. 

\subsection{System architecture}
Another important aspect is the system architecture for the playing robot. In our experiments we have used~\gls{ros} as the main programming framework. This development environment turned out to be very useful specially to what concerns the reuse of important pieces of software. For instance, instead of redesign the  mapping acquisition and SLAM software from scratch, we have saved time by reusing the ones available in~\gls{ros}: \textit{gmapping} and \textit{amcl}. 

Following software engineering practices, a possible direction for research is to develop and test a system architecture that is able to optimally place the different components and demands of a~\gls{pirg} system towards easy development. The development of best practice and design pattern to increase the code maintainability and interpretation would be useful specially to what concerns the reduction of the time to deploy the application.

\section{Final considerations}
On our work we have contributed with some new insights for the development of~\gls{pirg}s. Instead than designing a new environment from scratch, %EWERTON: I do not understand this (preceding) sentence we did design the environment from scratch. RoboTower 2.0 is different from RoboTower.
we have tested some models for taking into account the player activities and classify them into meaningful profiles from data. In the process, we have also contributed in showing how a popular discrete model from Data mining (the~\gls{lda} model) could be used in our game scenario. Being such model discrete, we provided a way to extend it to the continuous setup through the use of~\gls{gaf} images and unsupervised methods for feature extraction.

Our research provided evidence that the game is well accepted by players, easy to understand and safe to play. Despite not being able to reject the null hypothesis regarding the difference in the median distribution of responses for the perception of deceptive movements in the control and test groups, the obtained results show that deception is an expected behavior, which foment new research in its applicability of as a mean to increase entertainment. It is clear that deception should be a well designed behavior and its interpretation by the players ought to be accurate. New research on this topic is expected to include a player attention model in order to determine when to trigger and maximize the chances of the motion being actually interpreted as a deceptive behavior. Our research has provided the ground environment from which to pursue new inquiries.

The research and development of~\gls{pirg}s is still in its infancy, since works in the area are few and still mostly independent. In our research, and specially for its future directions, the use of~\gls{ml} algorithms turned out to be useful since data analysis was a required component for different platform tasks, from player tracking to player modeling, including navigation. The use of~\gls{ros} turned out to be of great importance in order to manage contemporaneously the different attributions needed in the application.

We believe we achieved the overall objective of our thesis, which was that of investigating different aspects of~\gls{pirg} design with the aim of obtaining a game that is entertaining. On this way, we have pointed out the different factors that can affect the perception of the robot as an agent not only rational (i.e., which has a goal), but as a player (that follows the rules of the game) ``smart enough'' to entertain. We hope that the game we developed for the necessary experimentation,\ie Robotower 2.0, will inspire the community, provide new useful insights and demonstrate the contribution of our present work. Full code of the application is available in our GitHub repository: \url{https://github.com/ewerlopes/phd_thesis.git}.

