%=============================================
% Commands
%---------------------------------------------

%---------------------------------------------
% Generic
%
\newcommand{\thesistitle}{Learning behaviors to optimize the player's experience in robogames}

\usepackage{xspace}
\DeclareRobustCommand{\eg}{e.g.,\@\xspace}
\DeclareRobustCommand{\ie}{i.e.,\@\xspace}
\DeclareRobustCommand{\wrt}{w.r.t.\@\xspace}


\newcommand{\Environment}{\mathcal{E}}
\newcommand{\Objective}{\mathcal{J}}
\newcommand{\statespace}{\mathcal{X}}
\newcommand{\observationspace}{\mathcal{O}}
\newcommand{\actionspace}{\mathcal{U}}
\newcommand{\Rmodel}{\mathcal{R}}
\newcommand{\Pmodel}{\mathcal{P}}
\newcommand{\Omodel}{\Omega}
\newcommand{\Imodel}{\mathcal{\iota}}
\newcommand{\Dataset}{\mathcal{D}}
\newcommand{\traj}{\tau}
\newcommand{\trajset}{\mathcal{T}}
\newcommand{\trajspace}{\mathbb{T}}

\newcommand{\gradient}[1]{\nabla_{#1}}
\newcommand{\Hessian}{\mathcal{H}}

\newcommand{\genericdist}{\mathcal{D}}


\DeclareMathOperator*{\argmax}{arg\,max}
\DeclareMathOperator*{\argmin}{arg\,min}

%---------------------------------------------
% Algorithms
%

\newcommand{\algorithmicinput}{\textbf{input}}
\newcommand{\algorithmicoutput}{\textbf{output}}
\newcommand{\INPUT}{\item[\algorithmicinput]}
\newcommand{\OUTPUT}{\item[\algorithmicoutput]}

%---------------------------------------------
% Math & Theorems
%

\newtheorem{Property}{Property}
\newtheorem{theorem}{Theorem}
\newtheorem{assumption}{Assumption}
\crefname{Property}{Property}{Properties}

%---------------------------------------------
% Hierarchical
%

\newcommand{\Cgraph}{\mathcal{G}}
\newcommand{\Cblocks}{B}
\newcommand{\Cblock}{b}
\newcommand{\CDedges}{D}
\newcommand{\CDedge}{d}
\newcommand{\CRedges}{C}
\newcommand{\CRedge}{c}
\newcommand{\CAedges}{A}
\newcommand{\CAedge}{a}

%---------------------------------------------
% IRL
%

%\newcommand{\SOMEIRL}{SOME-IRL\xspace}
%\newcommand{\GIRL}[1][]{GIRL}
%\newcommand{\PGIRL}[1][]{PGIRL}

\newcommand{\Rparams}{\boldsymbol{\omega}}
\newcommand{\Hparams}{\boldsymbol{\rho}}
\newcommand{\Pparams}{\boldsymbol{\theta}}
\newcommand{\PPspace}{\Theta}

%---------------------------------------------
% Figures
%
\newlength\figureheight 
\newlength\figurewidth 

%---------------------------------------------
% Glossary style
%
\setglossarystyle{altlist}

% table icons
%%%%% marks %%%%%

\newcommand{\mysquare}[1][black]{\small\textcolor{#1}{\ensuremath\blacksquare}}
\newcommand{\mycirc}[1][black]{\small\textcolor{#1}{\ensuremath\bullet}}
\newcommand{\mylozenge}[1][black]{\small\textcolor{#1}{\ensuremath\blacklozenge}}
\newcommand{\mytriangle}[1][red]{\small\textcolor{#1}{\ensuremath\blacktriangle}}
\newcommand{\mydtriangle}[1][black]{\small\textcolor{#1}{\ensuremath\blacktriangledown}}
\newcommand{\mystar}[1][black]{\Large\textcolor{#1}{\ensuremath\star}} %% or \bigstar
%%%%%%%%%%%%%%%%%%%%%%

%%%%% \norm command
\newcommand\norm[1]{\left\lVert#1\right\rVert}

%---------------------------------------------
% Listing style
%
\lstdefinestyle{customc}{
    frame=tb, % draw a frame at the top and bottom of the code block
    tabsize=4, % tab space width
    showstringspaces=false, % don't mark spaces in strings
    numbers=left, % display line numbers on the left
    xleftmargin=2em,
    framexleftmargin=1.5em,
    commentstyle=\color{green}, % comment color
    keywordstyle=\color{blue}, % keyword color
    stringstyle=\color{red} % string color
}

\lstset{escapechar=@,style=customc}


